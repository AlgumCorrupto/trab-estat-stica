\documentclass{article}

\usepackage[left=3cm, right=2cm, top=3cm, bottom=2cm, margin=1.0cm]{geometry}
\usepackage[12pt]{extsizes}
\usepackage[brazil]{babel}
\usepackage{graphicx}
\usepackage{hyperref}

\title{Análise sobre o Enem 2024}

\author{Paulo Artur Villaça}

\date{Dezembro, 2025}

\begin{document}

\maketitle

\section*{Introdução}

Este documento tem como objetivo apresentar uma análise do desempenho dos alunos de Campos dos Goytacazes no Exame Nacional do Ensino Médio (Enem) de 2024. O foco do trabalho é compreender aspectos quantitativos relacionados à participação e ao desempenho dos inscritos, com ênfase particular na prova de Matemática.

A análise concentra-se nos seguintes pontos:

\begin{itemize}
\item Presença dos alunos nos dias de prova;
\item Quantidade de acertos na prova de Matemática;
\item Notas obtidas na prova de Matemática.
\end{itemize}

A análise referente à quantidade de acertos e às notas é realizada a partir de uma amostra composta apenas pelos inscritos que estiveram presentes tanto no primeiro quanto no segundo dia do exame. Já a análise de presença considera o conjunto total de inscritos no município.

Dessa forma, a leitura deste artigo é significativamente enriquecida quando acompanhada da consulta direta ao \textit{Notebook} utilizado na análise, no qual estão detalhados os procedimentos computacionais, as visualizações gráficas e as etapas intermediárias dos cálculos. O \textit{Notebook} que contém a análise completa está disponível em \url{https://github.com/AlgumCorrupto/trab-estat-stica/blob/master/enem.ipynb}. Ressalta-se que a consulta a esse documento é indispensável, pois nele encontram-se todos os dados utilizados ao longo da análise.

\pagebreak

\section*{Análise}

\subsection*{Presença}

\begin{figure}[h]
    \centering
    \includegraphics[width=0.6\textwidth]{presencas.png}
    \caption{x = provas, y = número de presença}
    \label{fig:presencas}
\end{figure}


Do total de 2188 pessoas inscritas para realizar o Enem em Campos dos Goytacazes, apenas 1520 estiveram presentes em ambos os dias de prova.
Isso implica que aproximadamente $30{,}53\%$ dos inscritos não compareceram aos dois dias do exame, evidenciando uma taxa de ausência considerável.

\pagebreak

\subsection*{Sobre as Notas}

As análises apresentadas nesta e na próxima subseção referem-se exclusivamente aos inscritos que realizaram as quatro provas objetivas do Enem 2024 (Linguagens, Ciências Humanas, Ciências da Natureza e Matemática), excluindo-se a Redação, e com ênfase especial na prova de Matemática.

A nota média obtida na prova de Matemática foi de $\approx 540$ pontos. Esse valor é consistente com o método de correção adotado pelo INEP, que leva em consideração a distribuição geral de acertos dos participantes por meio da Teoria de Resposta ao Item (TRI). A mediana observada é ainda mais próxima do centro da escala, situando-se em torno de 525.5 pontos. A moda dessa distribuição localiza-se em 448 pontos.

No entanto, esse mesmo padrão de centralidade não se mantém quando se analisam os quartis da distribuição:
\begin{itemize}
    \item Q1 = 412.75
    \item Q3 = 629.0
\end{itemize}

\begin{figure}[h]
    \centering
    \includegraphics[width=0.6\textwidth]{poligono-notas.png}
    \caption{x = nota, y = frequência absoluta}
    \label{fig:poligono-notas}
\end{figure}

A amplitude das notas foi de 552.0 pontos, com a menor nota registrada sendo 373 e a maior, 925 pontos. O coeficiente de variação foi de aproximadamente $20{,}54\%$, indicando uma dispersão moderada dos dados. O desvio padrão observado foi de cerca de $110{,}97$ pontos em relação à média.

A distribuição das notas sugere um comportamento que não se aproxima perfeitamente de uma distribuição normal, apresentando assimetrias perceptíveis. O coeficiente de assimetria foi de aproximadamente $0{,}46$, indicando uma assimetria à direita. Além disso, o grau de curtose foi de aproximadamente $-0{,}61$, caracterizando uma distribuição platicúrtica.

Uma peculiaridade observada durante a análise é que a maior nota registrada na prova de Linguagens foi de aproximadamente 720 pontos, valor consideravelmente inferior ao máximo observado na prova de Matemática.

\pagebreak

\subsection*{Sobre os Acertos}

Em média, os inscritos acertaram 13 questões na prova de Matemática, sendo a quantidade de acertos mais frequente igual a 10 questões,enquanto a mediana da distribuição foi de 12 acertos.

Observa-se que a menor quantidade de acertos registrada não foi zero, mas sim 3 questões, enquanto o maior número de acertos atingiu 43 das 45 questões da prova, resultando em uma amplitude de 40 questões. O coeficiente de variação dessa variável foi de aproximadamente 48.4\%, o que indica uma alta dispersão nos resultados, com um desvio padrão de aproximadamente $6{,}45$ questões em relação à média.

$75\%$ dos inscritos acertaram menos de 16 questões na prova de Matemática, enquanto apenas $10\%$ dos participantes acertaram mais de 22 questões.

Seu coeficiente de assimetria é um valor positivo $\approx 1{,}30$, simbolizando uma distribuição assimétrica bastante acentuada para direita e com curtose de $\approx 2{,}18$, sendo classificada como uma distribuição leptocúrtica.

O polígono de frequência da distribuição dos acertos, construído com 12 classes, é apresentado a seguir.

\begin{figure}[h]
    \centering
    \includegraphics[width=0.6\textwidth]{poligono-acertos.png}
    \caption{x = quantidade de acertos, y = frequência absoluta}
    \label{fig:poligono-acertos}
\end{figure}

Observa-se ainda a inclusão das distribuições referentes às demais provas. Nota-se que o maior número de acertos na prova de Linguagens é inferior a 25 questões, o que sugere que um participante pode atingir uma nota em torno de 700 pontos mesmo acertando menos da metade do total de questões dessa prova.


\pagebreak

\section*{Referências}

\begin{itemize}
  \item \url{https://www.gov.br/inep/pt-br/acesso-a-informacao/dados-abertos/microdados/enem}
  -- Microdados do Enem

  \item \url{https://www.gov.br/inep/pt-br/areas-de-atuacao/avaliacao-e-exames-educacionais/enem}
  -- Informações institucionais sobre o Enem

  \item \url{https://download.inep.gov.br/publicacoes/institucionais/avaliacoes_e_exames_da_educacao_basica/entenda_a_sua_nota_no_enem_guia_do_participante.pdf}
  -- Guia explicativo oficial sobre cálculo de notas e TRI

  \item \url{https://github.com/AlgumCorrupto/trab-estat-stica}
  -- Análise pessoal
\end{itemize}

\end{document}
