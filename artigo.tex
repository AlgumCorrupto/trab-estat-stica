\documentclass{article}

\title{Análise sobre Enem 2024}

\author{Paulo Artur Villaça}

\date{Dezembro, 2025}

\begin{document}

\section*{Introdução}

Esse documento tem como objetivo oferecer uma análise sobre a peformace dos alunos de Campos dos Goytacazes da prova do Enem em 2024.
O documento tem como objetivo analisar.

- A quantidade de acertos dos alunos 

- A notas dos alunos

- Presença dos alunos

A análise da quantidade de acertos e notas será feita por uma amostra composta pelos alunos que fizeram as provas do primeiro e segundo dia, a análise sobre a presença será feita na população que se inscreveu no Enem de 2024 e fez a prova em campos.

\section*{Análise}

\subsection*{Presença}

2188 Inscritos fizeram a prova em Campos dos Goytacazes enqunato 1520 dos inscritos fizeram ambas a provas, isso significa que aproximadamente 30.530\% dos alunos não fizeram as duas provas.  

\subsection*{Sobre as Notas}

As próximas duas seções do artigo analisam a peformace dos inscritos que fizeram todas as 4 provas do Enem 2024 (linguagens, ciências humanas, ciências naturais e matemática), com exceção da Redação e com ênfase na prova de matemática.

Em média a notas da prova de matemática dos inscritos foi 540 pontos, o que é esperado pois a INEP já leva em consideração a média de acertos dos alunos em consideração para determinar a sua nota, sua mediana é ainda mais próximo da metade da nota máxima possível, valendo aproximadamente 525.5.

Sua amplitude é 552.0, com a menor nota sendo 373 e maior nota 925, com variação de aproximadamente 20.54\\% em relação a média.

A sua distribuição aparenta não ser uma distribuição normal perfeita.

% TODO: Colocar a foto da distribuição normal aqui

Uma peculiaridade encontrada é que o maior nota encontrada na prova de Linguagens durante a análise foi de 720 pontos.

\subsection*{Sobre os Acertos}

Em média, os inscritos acertaram 13 questões na prova de matemática, com maior frequência de acertos sendo 10 questões.

Incrivelmente a menor quantidade de acertos não foi 0, mas sim 3 questões, com a maior quantidade de acertos sendo 43 de 45 questões. Em média, essa quantidade varia 48.4\%, que significa que essa sequência de dados tem uma alta dispersão.

O histograma da distribuição com 12 classes, temos esse plot.

% TODO: Colocar foto da distrubuição normal aqui

Note-se que também foi incluido a distribuição das outras provas. A maior quantidade de acetos da prova de linguagens é menor que 25 acertos, isso significa que para um inscrito acertar 700 pontos, ele precisa acertar menos da metade do total de questões.

\end{document}

